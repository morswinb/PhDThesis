\subsubsection{Grids}
All grids assume spacing of $N$ number of points in an interval of $\tilde x \in [0,1-\delta]$ with some small value of delta $\delta$.

\paragraph{Regular uniform grid}
Regular uniform grid 
\begin{equation}
\tilde x_k = (1-\delta)k/N,\quad k=0,1,...,N
\end{equation}
witch divides the interval into $N$ equally spaced points.
\paragraph{Exponential grid}
Modification of the exponential law:
\begin{equation}
\tilde x_k =\tanh(ak), \quad k=0,1,...,N
\end{equation}
where parameter $a$ should be computed form the condition:
\begin{equation}\label{grid_condition_1}
\tilde x_N=1-\delta
\end{equation}
Such grid has a benefit of being denser in the proximity of crack tip where the solution for the problem is harder to obtain, therefore allows for greater accuracy.
\paragraph{Mixed grid}
A mixture of both uniform and exponential grid defined as:
 \begin{equation}
   \tilde x_k = \left\{
     \begin{array}{cl}
       bj/M, & k=0,1,...,M<N\\
       1-Ce^{-ak}, & j=M+1,M+2,...,N,
     \end{array}
   \right.
\end{equation} 
Which is a mixture of the uniform grid in prescribed in the interval $[0,b],\; (b<1-\delta)$ and the exponential in a vicinity of the crack tip $[b,1-\delta]$. The mesh parameters $(a,b,C,M)$ are related via equation \eqref{grid_condition_1} and 
\begin{equation}
\tilde x_{M-}=\tilde x_{M+},\quad \Delta \tilde x_{M}=\Delta \tilde x_{M+1}
\end{equation} 
where $\tilde x_{M-}$ and $\tilde x_{M+}$ are the matching points of those two different subdivisions, $\Delta \tilde x_k = \tilde x_k -\tilde x_{k-1}$. Moreover one of the parameters ($b$ or $M$) is left undefined and can be used for adaptivity.

\subsubsection{Test functions and approximation schemes}
As the test function the following function of shape of a solution of hydraulic fracture problem was choosen:

\begin{equation}
w(x)=A(1-x)^{\alpha}+e^{Cx}(1-x)^{\frac{4}{3}}
\end{equation} 

Where $A$ and $C$ are constants that were arbitrary set to equal 1 and $\alpha$ corresponds to elasticy of the fluid (???). The values of $\alpha=\frac{1}{3}$ for newtonian fluid and $\alpha=\frac{2}{3}$ for non newtonian fluid.

There are 3 approximation schemes considered for finding the value of first and second derevative of the function $w(x)$, at the points $x_k,\; k=2,3,...,N-1$. The end points of the intervals are omited as these the derevatives at this points would be set by the boundary conditions in furthere computations.

\paragraph{Finite Differences}
The standard centrial finite difference scheme for the first derevative and its recursive(hum ?) application gives:
\begin{equation}
\begin{split}\label{FD}
\frac{\partial w}{\partial \tilde x}|_{x_k}=&\frac{1}{2}
\left(\frac{\Delta w_{k+1}}{\Delta \tilde x_{k+1}}
+\frac{\Delta w_{k}}{\Delta \tilde x_{k}}\right)
\\
\frac{\partial^2 w}{\partial \tilde x^2}|_{x_k}=&\frac{1}{2}
\left(\frac{1}{\Delta \tilde x_{k+1}}+\frac{1}{\Delta \tilde x_{k}}\right)
\left(\frac{\Delta w_{k+1}}{\Delta \tilde x_{k+1}}-\frac{\Delta w_{k}}{\Delta \tilde x_{k}}\right)
\end{split}
\end{equation}

Or

\begin{equation}
\frac{\partial w}{\partial \tilde x}|_{x_k}=a_k w_{k-1}+b_k w_{k}+c_k w_{k+1}
\end{equation}

\begin{equation}
a_k=-\frac{1}{2(x_k-x_{k-1})}
\end{equation}

\begin{equation}
b_k=\frac{1}{2}\left(\frac{1}{x_k-x_{k-1}}-\frac{1}{x_{k+1}-x_k}\right)
\end{equation}

\begin{equation}
c_k=\frac{1}{2(x_{k+1}-x_k)}
\end{equation}

And

\begin{equation}
\frac{\partial^2 w}{\partial \tilde x^2}|_{x_k}=a_k w_{k-1}+b_k w_{k}+c_k w_{k+1}
\end{equation}

\begin{equation}
a_k=\frac{1}{2}\left(\frac{1}{x_{k+1}-x_k}+\frac{1}{x_k-x_{k-1}}\right)\frac{1}{x_k-x_{k-1}}
\end{equation}

\begin{equation}
b_k=-\frac{1}{2}\left(\frac{1}{x_{k+1}-x_k}+\frac{1}{x_k-x_{k-1}}\right)^2
\end{equation}

\begin{equation}
c_k=\frac{1}{2}\left(\frac{1}{x_{k+1}-x_k}+\frac{1}{x_k-x_{k-1}}\right)\frac{1}{x_{k+1}-x_k}
\end{equation}

\paragraph{Quadratic Polynomial interpolation}
Interpolating a quadratic polynomial in form of $f(\tilde x)=a(\tilde x-\tilde x_k)^2+b(\tilde x-\tilde x_k)+c$. This interpolated on points $\tilde x_{k-1},\tilde x_{k},\tilde x_{k+1}$ would give a system: 
\begin{equation}
\begin{split}
a(\tilde x_{k-1}-\tilde x_k)^2+b(\tilde x_{k-1}-\tilde x_k)&=w_{k-1}-w_{k}
\\
a(\tilde x_{k+1}-\tilde x_k)^2+b(\tilde x_{k+1}-\tilde x_k)&=w_{k+1}-w_{k}
\end{split}
\end{equation} 
And such so is solved by
\begin{equation}
\begin{split}
a=&\left(\frac{\Delta w_{k+1}}{\Delta \tilde x_{k+1}}\Delta \tilde x_{k}-\Delta w_{k}\right)
\big(\Delta \tilde x_{k}\left(\Delta \tilde x_{k}+\Delta \tilde x_{k+1}\right)\big)^{-1}
\\
b=&\frac{\Delta w_{k+1}}{\Delta \tilde x_{k+1}}-a\Delta \tilde x_{k+1}
\end{split}
\end{equation} 
And then so we can say find that the values of $f'(\tilde x_k)$ and of $f''(\tilde x_k)$ are approximated by:
\begin{equation}
\begin{split}\label{FD}
\frac{\partial w}{\partial \tilde x}|_{x_k}= &\frac{1}{\Delta \tilde x_{k+1}+\Delta \tilde x_{k}}
\left(\frac{\Delta w_{k+1}\Delta \tilde x_{k}}{\Delta \tilde x_{k+1}}
+\frac{\Delta w_{k}\Delta \tilde x_{k+1}}{\Delta \tilde x_{k}}\right)
\\
\frac{\partial^2 w}{\partial \tilde x^2}|_{x_k}=&\frac{2}{\Delta \tilde x_{k+1}+\Delta \tilde x_{k}}\left(\frac{\Delta w_{k+1}}{\Delta \tilde x_{k+1}}
-\frac{\Delta w_{k}}{\Delta \tilde x_{k}}\right)
\end{split}
\end{equation}

Or

\begin{equation}
\frac{\partial w}{\partial \tilde x}|_{x_k}=a_k w_{k-1}+b_k w_{k}+c_k w_{k+1}
\end{equation}

\begin{equation}
a_k=-\frac{1}{x_{k+1}-x_{k-1}}\left(\frac{x_{k+1}-x_{k}}{x_{k}-x_{k-1}}\right)
\end{equation}

\begin{equation}
b_k=-\frac{1}{x_{k+1}-x_{k-1}}\left(\frac{x_{k+1}-x_{k}}{x_{k}-x_{k-1}}-\frac{x_k-x_{k-1}}{x_{k+1}-x_k}\right)
\end{equation}

\begin{equation}
c_k=-=\frac{1}{x_{k+1}-x_{k-1}}\left(\frac{x_k-x_{k-1}}{x_{k+1}-x_k}\right)
\end{equation}

And

\begin{equation}
\frac{\partial^2 w}{\partial \tilde x^2}|_{x_k}=a_k w_{k-1}+b_k w_{k}+c_k w_{k+1}
\end{equation}

\begin{equation}
a_k=\frac{2}{x_{k+1}-x_{k-1}}\left(\frac{1}{x_{k}-x_{k-1}}\right)
\end{equation}

\begin{equation}
b_k=-\frac{2}{x_{k+1}-x_{k-1}}\left(\frac{1}{x_{k+1}-x_{k}}+\frac{1}{x_{k}-x_{k-1}}\right)
\end{equation}

\begin{equation}
c_k=\frac{2}{x_{k+1}-x_{k-1}}\left(\frac{1}{x_{k+1}-x_{k}}\right)
\end{equation}

\paragraph{Cubic spline interpolation}

spline po przedziale z matlaba, duzo do opisania...

