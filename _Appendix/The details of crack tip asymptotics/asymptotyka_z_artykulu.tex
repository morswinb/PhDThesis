\section{Asymptotics of the solutions for different leak-off functions} \label{app:C}


Asymptotic expansion for the crack opening and the fluid velocity near the crack tip in the normalised variables (\ref{dimensionless}) can be written in the following general forms:
\begin{equation}\label{w_asym}
w(t, x)=\sum_{j=0}^Nw_j(t)(1-x)^{\alpha_j}+O((1- x)^{\varrho_w}), \quad x\rightarrow 1,
\end{equation}
and
\begin{equation}\label{v_asym}
V(t, x)=\sum_{j=0}^NV_j(t)(1-x)^{\beta_j}+O((1- x)^{\varrho_V}),\quad  x\rightarrow 1,
\end{equation}
with $\varrho_w>\alpha_n$, $\varrho_V>\beta_n$, $\alpha_0=1/3$, $\beta_0=0$ and some increasing sequences $\alpha_0,\alpha_1,\ldots,\alpha_n$ and $\beta_0,\beta_1,\ldots,\beta_n$. Note that the asymptotics are related to each other by the speed equation \eqref{norm_speed} and thus, regardless of the chosen leak-off function, we can write
\begin{equation}\label{v_asym_1}
\sum_{j=0}^NV_j(t)(1-x)^{\beta_j}+\ldots=
\end{equation}
\[
\frac{1}{3L(t)}\sum_{k=0}^N\sum_{m=0}^N\sum_{j=0}^N\alpha_jw_j(t)
w_m(t)w_k(t)(1-x)^{\alpha_j+\alpha_m+\alpha_k-1}.
\]
In line with the discussion after equation (\ref{V_asym_0}),
we are interested only in the terms such that $\beta_j\le1$, restricting ourselves to the smallest $\varrho_V>1$,
since the values of $\beta_j$ are combinations of a sum of three consequent components of the exponents $\alpha_j$.
However, since $\alpha_0$ is known ($\alpha_0=1/3$), one can write (compare with (\ref{V_asym_0_coefs})):
\begin{align}
& V_{0}(t)=\frac{1}{3L(t)}w_{0}^{3}(t)\label{v_0}, \\
& V_{1}(t)=\frac{1}{L(t)}\left(\alpha_1+\frac{2}{3}\right) w_{0}^{2}(t) w_{1}(t),\quad \beta_1=\alpha_1-\frac{1}{3}.\label{v_1}
\end{align}

To continue the process one now needs to compute the value of the exponent $\alpha_1$ as it is not clear 
a priori which value determining the next exponent $\beta_2=\min\{2/3+\alpha_2,1/3+2\alpha_1\}$ is larger. To do so
let us rewrite the continuity equation (\ref{norm_continuity}) in the form:
\begin{equation}
\label{norm_continuity_as}
\frac{\partial w}{\partial t}+\frac{V_0(t)}{L( t)}(1-x)\frac{\partial w}{\partial x}=\frac{1}{L( t)}\frac{\partial \big(w(V_0-V)\big)}{\partial x}-q_l(t,x).
\end{equation}
Here, the terms on the left-hand side of the equation are always bounded near the crack tip, while those on the right-hand side can behave differently depending on the chosen leak-off function.


\noindent Consider the following three cases of $q_l$ behaviour.

(\emph{i}) Assume first that
\[
q_l(t,x)=o\big(w(t,x)\big),\quad x\to1.
\]
This case naturally includes the impermeable rock formation. Analysing the leading order terms in the equation (\ref{norm_continuity_as}), it is clear that $w(V_0-V)=O((1-x)^{4/3})$, as $x\to1$. This, in turn, is only possible for $\beta_1=1$ and, therefore, $\alpha_1=4/3$. Finally, comparing the left-hand side and the right-hand side of the equation we obtain:
\begin{equation}
w_{0}'(t)= \frac{w_{0}(t)}{3L(t)} \big(V_0(t)+4V_1(t)\big),\quad V_{1}(t)=\frac{2}{L(t)}w_{0}^{2}(t) w_{1}(t).
\label{sp_1}
\end{equation}
This case has been considered in \cite{Linkov_4} and \cite{MWL}.

\vspace{2mm}


(\emph{ii})
If we assume that the leak-off function is estimated by the solution as  $O\big(w(t,x)\big)$, or equivalently;
\[
q_l(t,x)\sim\Upsilon(t)w_0(t)(1-x)^{1/3},\quad x\to1,
\]
then the previous results related to the values of $\alpha_1$ and $\beta_1$ and, therefore, the equation (\ref{sp_1})$_2$ remain the same, while the first one changes to
\begin{equation}
w_{0}'(t)= \frac{1}{3L(t)} w_{0}(t)\big(V_0(t)+4V_1(t)\big)-\Upsilon(t)w_0(t).
\label{sp_2}
\end{equation}
This case corresponds to (\ref{norm_leak_off_1})$_3$ when $C_{32}=0$ and $\Upsilon(t)=kC_{31}(t)$.

\vspace{2mm}


(\emph{iii})
The leak-off function in a general form:
\[
q_l(t,x)=\Phi(t)(1-x)^{\theta}+o((1-x)^{1/3}),\quad x\to1,
\]
where $-1/2\le \theta<1/3$. Here, one can conclude that $w(V_0-V)=O((1-x)^{1+\theta})$, as $x\to1$ or equivalently, $\beta_1=\theta+2/3$, and $\alpha_1=1+\theta$. Moreover, in this case:
\begin{equation}
(1+\theta)w_0V_1=L(t)\Phi(t), \quad V_{1}(t)=\frac{1}{L(t)}\left(\theta+\frac{4}{3}\right) w_{0}^{2}(t) w_{1}(t),
\label{sp_3}
\end{equation}
and, thus
\begin{equation}
w_{1}(t)=\frac{3L^2(t)\Phi(t)}{(4+3\theta)(1+\theta)w_{0}^{3}(t)}.
\label{sp_3a}
\end{equation}

Note, that as one would expect, the particle velocity function is not smooth in this case near the crack tip, its derivative is unbounded and exhibits the following behaviour:
\[
\frac{\partial V}{\partial x}=O\big((1-x)^{\theta-1/3}\big), \quad x\to1.
\]
To formulate the equation similar to (\ref{sp_1})$_1$ or (\ref{sp_2}), one needs to continue asymptotic analysis of the equation (\ref{norm_continuity_as}) incorporating the available information.
Apart from the fact that the analysis can be done in the general case, we restrict ourselves only to three variants used from the beginning (compare (\ref{carter})), respectively:
$\theta=0$, $\theta=1/3-1/2=-1/6$ and $\theta=-1/2$.





When $\theta=0$, $\alpha_1=1$ and $\beta_1=2/3$, returning to the equation (\ref{v_asym_1}), one concludes that $\beta_2>1$ and, therefore,
\begin{equation}
w_{0}'(t)= \frac{1}{3L(t)} w_0(t)V_0(t).
\label{sp_4}
\end{equation}
This case corresponds to (\ref{norm_leak_off_1})$_3$ when $\Phi(t)=C_3^{(2)}(t)w_0(t)$ and $C_3^{(1)}=0$.


If $\theta=-1/6$, then $\alpha_1=5/6$ and $\beta_1=1/2$. In this case the function $\Phi(t)$ can be written as $\Phi(t)=C_2 D(t)w_0(t)$ (compare to (\ref{norm_leak_off_1})$_2$) 
and again equation (\ref{v_asym_1}) gives $\beta_2>1$, while equation (\ref{norm_continuity_as}) leads to
\begin{equation}
w_{0}'(t)= \frac{1}{3L(t)} \big(w_0(t)V_0(t)+4w_1(t)V_1(t)\big).
\label{sp_5}
\end{equation}
Summarizing, in both mentioned above cases, there exists a single term in asymptotics of the particle velocity which has singular derivative near the crack tip. Moreover, those terms ($w_1$ and $V_1$, respectively) are fully defined by the leak-off function $\Phi(t)$ and the coefficient $w_0$ in front of the leading term for the crack opening in (\ref{sp_3a}) and (\ref{sp_3})$_1$.

The situation changes dramatically when $\theta=-1/2$ (Carter law). We now have $\alpha_1=1/2$ and $\beta_1=1/6$ and $\Phi(t)=C_1D(t)$. In this case, however, $\beta_2<1$ and we need to continue the asymptotic analysis further to evaluate all terms of the particle velocity which exhibit non-smooth  behaviour near the crack tip. We omit the details of the derivation, presenting only the final result in a compact form.
The first six exponents in the asymptotic expansions (\ref{w_asym}) and (\ref{v_asym}), that introduce the singularity of $w_x$, are:
\[
\alpha_j=\frac{1}{2}+\frac{j}{6},\quad \beta_j=\frac{j}{6},\quad j=1,2,\ldots,6.
\]
\[
w_j(t)=\kappa_j\frac{\Phi^j(t)L^{2j}(t)}{w_0^{4j-1}(t)},\quad V_j(t)=\psi_j\frac{\Phi^j(t)L^{2j-1}(t)}{w_0^{4j-3}(t)},
\]
where $j=1,2,\ldots,5$ and
\[
\begin{array}{l}
\kappa_1=\frac{12}{7},\quad \psi_1=2,\quad \kappa_2=-\frac{270}{49},\quad \psi_2=-\frac{24}{7},
\\[4mm]
\kappa_3=\frac{9768}{343}, \,\, \psi_3=\frac{828}{49},\,\, \kappa_4=-\frac{2097252}{12005},\,\, \psi_4=-\frac{5136}{49},
\\[4mm]
\kappa_5=\frac{1081254096}{924385},\quad \psi_5=\frac{1234512}{1715}.
\end{array}
\]